\documentclass[a4paper]{report} % o article, book, ...
% packages...
\usepackage[utf8]{inputenc}
\usepackage[english,italian]{babel}
\usepackage[hyphens]{url}
% Per generare il file PDF aderente alle specifiche PDF/A-1b. Verificarne poi la validità.
%\usepackage[a-1b]{pdfx}
\usepackage{hyperref}
\usepackage{graphicx}
% Per inserire testo a caso in attesa di realizzare i capitoli
\usepackage{lipsum}
%%%%%%%%%%%%%%%%%%%%%%%%%%%%%%%%%%%%%%%%%%%%%%%%%%%%%
\begin{document}
% Frontespizio
\begin{titlepage}
\begin{center}
\includegraphics[width=\textwidth]{Logo.jpg}\\
{\large{\bf Corso di Laurea Triennale in Informatica}}
\end{center}
\vspace{12mm}
\begin{center}
{\huge{\bf Sperimentazione sul campo}}\\
\vspace{4mm}
{\huge{\bf di rete LPWAN basata}}\\
\vspace{4mm}
{\huge{\bf sul protocollo LoRaWAN}}\\
\end{center}
\vspace{12mm}
\begin{flushright}
{\large{\bf Tesi di Laurea di:}}\\
{\large{\bf Matteo Curti}}\\
{\large{\bf Matr. 802690}}\\
\end{flushright}
\vspace{4mm}
\begin{flushleft}
{\large{\bf Relatore:}}\\
{\large{\bf Andrea Trentini}}\\
\vspace{4mm}
%{\large{\bf Correlatore:}}\\
%{\large{\bf CORREL}}\\
\end{flushleft}
\vspace{12mm}
\begin{center}
{\large{\bf Anno Accademico 2017/2018}}
\end{center}
\end{titlepage}
\tableofcontents
% o sections (dipende dal documentclass)
\chapter{Cap1}
\section{Sensori e trasduttori}
\subsection{I sensori}
I sensori, in un sistema di acquisizione dati, sono le interfacce fisiche tra i dispositivi di elaborazione dei segnali elettrici acquisiti in input e le corrispondenti grandezze fisiche da monitorare in funzione del tempo. 

Un sensore è un dispositivo con la funzione di fornire in uscita un segnale elettrico, sotto forma di tensione o di corrente, del valore teoricamente proporzionale a quello della grandezza fisica in input che si desidera acquisire. 

Per ottenere un segnale elettrico in uscita di un valore linearmente dipendente da una qualsiasi grandezza fisica (temperatura, pressione, forza, flusso luminoso, ecc..) in input, si utilizza il sensore, il cui funzionamento si basa sulla variabilità di una grandezza fisica elettrica, per esempio resistenza, capacità o induttanza, in funzione della grandezza fisica da misurare.

Si definisce caratteristica di trasferimento la curva che rappresenta l'andamento del segnale elettrico generato dal sensore in output in funzione della grandezza fisica che si considera.
La caratteristica di trasferimento è il legame che è presente tra la variabile da misurare (ingresso) e il segnale elettrico di uscita del trasduttore. 
Nella pratica accade molto raramente che tale caratteristica di trasferimento corrisponda, anche approssimativamente, alla legge di proporzionalità diretta: 
\begin{equation}
V = k G
\end{equation}
Oppure:
\begin{equation}
I = k G
\end{equation}
dove V è la tensione, mentre I è la corrente fornita dal sensore, k è una costante di taratura e G è la grandezza fisica da acquisire in input.

In modo generale la caratteristica di trasferimento si allontana, in modo più o meno pronunciato, dal modello ideale a causa di alcuni errori come l'errore di offset e l'errore di linearità. I trasduttori la cui caratteristica è una retta sono detti lineari. Il funzionamento ottimale di un trasduttore è definito da una caratteristica lineare. La linearità è il parametro che evidenzia la deviazione tra la retta (caratteristica teorica) e la curva reale. La non linearità è il valore massimo della deviazione rispetto alla curva teorica in valore assoluto riferito al valore massimo del segnale di uscita. Un sensore è buono quando la sua non linearità è inferiore allo 0.1%.

L'errore di offset è presente perché il segnale elettrico fornito in uscita è diverso da zero anche quando la grandezza fisica ha valore nullo. Questo avviene quando la retta non passa per l'origine la variabile d'uscita è diversa da zero in corrispondenza del valore nullo della variabile di ingresso. Supponendo che la caratteristica sia lineare:
\begin{equation}
V = V_{off} + k G
\end{equation}
dove $V_{off}$ è la tensione di offset, k è una costante di taratura e G è la grandezza fisica da acquisire in input. Si definisce offset il valore non nullo della variabile di uscita corrispondente al valore nullo della variabile d' ingresso.
L'errore di linearità è il massimo scarto, in valore assoluto, nel range nominale di funzionamento del sensore tra la caratteristica reale e la caratteristica ideale, dove per caratteristica ideale si intende la caratteristica lineare V, passante per gli estremi della caratteristica reale, nei punti (0, Voff) e (Fmax, Vmax), dove Fmax e Vmax sono rispettivamente il valore massimo della grandezza fisica G ed il valore massimo della tensione V nel range nominale. Il range di funzionamento è l'intervallo dei valori che può assumere la grandezza fisica che deve essere trasdotta. Appena la grandezza fisica esce dal range il trasduttore non funziona più, e ritorna a lavorare appena rientra nell'intervallo. Il range di ingresso (o campo di ingresso) definisce i limiti entro cui può variare il segnale in ingresso; mentre il range di uscita (o campo di uscita) definisce i limiti entro cui può variare il segnale in uscita. 

\subsection{Tipi di sensori}
I sensori si distinguono in base al principio di funzionamento:
\begin{enumerate}
\item sensori attivi, se forniscono direttamente un segnale elettrico, in tensione o in corrente, in funzione della grandezza fisica;
\item sensori passivi, se si utilizza la variabilità di una grandezza elettrica, per es. la resistenza, la capacità o l'induttanza, in funzione della grandezza fisica, per ottenere una tensione o una corrente, inserendo il sensore in un circuito di misura.  
\end{enumerate}


I sensori si distinguono in base al principio fisico utilizzato:
\begin{enumerate}
\item Resistivi: si basano sulla variabilità della resistenza elettrica in funzione degli sforzi meccanici, della temperatura, del campo magnetico, dell'intensità di illuminamento;
\item Piezoelettrici: si basano sul campo elettrico generato nei cristalli piezoelettrici da sforzi meccanici di trazione, compressione o taglio;
\item Termoelettrici: si basano sulle forze elettromotrici termoelettriche generate per effetto Seebeck da giunzioni metalliche mantenute a temperature diverse dette termocoppie;
\item Fotovoltaici: si basano sulle forze elettromotrici generate per effetto fotovoltaico da una giunzione semiconduttrice PN, colpita da radiazioni infrarosse, visibili, ultraviolette, X,        $ \gamma $ o da particelle cariche;
\item Fotoelettrici: si basano sulla fotocorrente che si ottiene, per effetto fotoelettrico, nelle celle fotoelettriche a vuoto, nei fotodiodi e nei fototransistor;
\item Ad effetto Hall: si basano sulle forze elettromotrici generate in particolari materiali, generalmente semiconduttori, percorsi da corrente e sottoposti a campi magnetici;
\item Capacitivi: si basano sulla variabilità della capacità elettrica di un condensatore in funzione dell'umidità o della costante dielettrica dell'isolante posto tra le armature;
\item Induttivi: si basano sulla variabilità dell'induttanza di un avvolgimento dotato di un nucleo ferromagnetico estraibile;
\item Elettromagnetici: si basano sulla legge dell'induzione elettromagnetica.
\item A riluttanza variabile: si basano sulla variabilità della riluttanza di un circuito magnetico dotato di parti mobili;
\item Potenziometrici: si basano sulla variabilità della tensione fornita da un partitore potenziometrico regolabile in funzione dello spostamento del cursore;
\item Piezoacustici e ultrasonici: si basano sulle forze elettromotrici piezoelettriche generate da particolari materiali ceramici sottoposti ad onde meccaniche;
\item Elettrochimici: si basano sui potenziali elettrochimici generati da celle elettrolitiche speciali, costituite da coppie di elettrodi sensibili a determinati ioni o radicali chimici, oppure sulla variabilità della conducibilità di particolari materiali in presenza di gas o vapori;
\item Biologici: si basano sulle forze elettromotrici generate da celle elettrolitiche speciali in presenza di enzimi o di altre biomolecole.
\end{enumerate}

\subsection{Classificazione dei sensori}
La classificazione dei sensori viene fatta in base alle caratteristiche fisiche:
\begin{enumerate}
\item resistivi: sfruttano la variazione della resistenza (fotoresistori, termoresistori, sensori di posizione);
\item capacitivi: sfruttano la variazione della capacità di un condensatore (sensori di umidità);
\item elettroacustici: convertono segnali sonori in grandezze elettriche (microfoni);
\item elettrodinamici: si basano sul principio della forza elettromotrice per misurare velocità (dinamo tachimetrica);
\item elettromagnetici: utilizzano il principio dell'induttanza elettrica per rilevare angoli di rotazione;
\item magnetostritivi: si fondano sul principio della permeabilità;
\item piezoelettrici: sfruttano l'originarsi di una polarizzazione elettrica su facce opposte di cristalli sottoposti a sollecitazioni (stress) fisiche;
\item semiconduttore: sfruttano le caratteristiche della giunzione dei semiconduttori (fotodiodi, fototransistor).
\end{enumerate}

\subsection{I trasduttori}
Il trasduttore è un dispositivo in grado di trasformare le variazioni di una grandezza fisica,
normalmente non elettrica, in un'altra grandezza, normalmente di natura elettrica (tensione, frequenza o corrente). E' composto da due parti: sensore e convertitore, un circuito elettronico che trasforma le variazioni di un parametro del sensore in una variazione di una grandezza elettrica.


\subsection{Tipi di trasduttori}
I trasduttori sono di due tipi:
\begin{enumerate}
\item Analogico: quando il suo segnale di uscita è una grandezza elettrica che varia in modo continuo mantenendo una doppia corrispondenza con il valore della grandezza misurata;
\item Digitale: quando il suo segnale di uscita è composto da uno o più segnali digitali che possono assumere ciascuno solo due livelli di tensione identificati come 0 e 1;
\end{enumerate}
Inoltre i trasduttori possono essere:
\begin{enumerate}
\item Attivi: Quando forniscono in uscita un segnale direttamente utilizzabile da circuiti di elaborazione senza nessun consumo di energia elettrica, è il caso delle celle fotovoltaiche e delle termocoppie;
\item Passivi: Sono quei trasduttori ai quali bisogna fornire energia elettrica perché la grandezza fisica d'uscita possa essere trasformata in una grandezza elettrica.
\end{enumerate}



\section{Parametri caratteristici}
\subsection{La sensibilità}
La sensibilità è il rapporto tra la variazione della grandezza di uscita e la variazione della grandezza d'ingresso che la provoca. 
\begin{equation}
S = \Delta U / \Delta I
\end{equation}
Più il coefficiente angolare della retta è elevato più il trasduttore è sensibile e minore sarà il range di funzionamento. Lo strumento risulterà essere molto sensibile quando a parità di grandezza di ingresso la grandezza di uscita è molto elevata.
Tempo di risposta è il tempo che il trasduttore impiega per raggiungere in uscita il valore di regime corrispondente al valore d'ingresso. 

\subsection{La risoluzione}
La risoluzione è il rapporto percentuale tra la minima variazione della grandezza di uscita in grado di essere rilevata e il valore massimo del fondo scala.
La risoluzione R esprime la variazione minima di uscita rispetto al fondo-scala:
\begin{equation}
R = \Delta Xout_{min} / \Delta Xout_{fondoscala}
\end{equation}

\subsection{La riproducibilità}
La riproducibilità è la capacità di un sensore di fornire sempre gli stessi valori di uscita in corrispondenza dell'ingresso. Vale a dire la costanza nel tempo delle caratteristiche del trasduttore (la sua resistenza all'invecchiamento).

\subsection{L'accuratezza}
Nel funzionamento reale il sensore descrive una caratteristica che si discosta dalla funzione ideale. Per tenere in conto la mancata accuratezza del dispositivo occorre misurare le deviazioni esistenti fra i valori reali e i valori ideali. Le caratteristiche di affidabilità in un sensore sono relazionate alla sua vita utile ed a possibili cause di mal funzionamento nel sistema in cui è inserito. Infatti  è la capacità del sensore di espletare la funzione per cui è stato costruito in condizioni prestabilite e per un tempo fissato; questo parametro è espresso in termini statistici come la probabilità che il dispositivo funzioni per un tempo o per un numero di cicli specificato. Solo di rado esso è specificato dai costruttori.

\subsection{Il drift}
Modifica temporale impredicibile delle caratteristiche del sensore. In pratica, la curva di risposta si modifica col tempo per cui, nella stima del misurando si introduce un errore che cambia (in genere cresce col tempo). Il drift definisce il tempo di vita della calibrazione del sensore, cioè dopo quanto tempo usare la stessa curva di risposta da luogo ad errori sul misurando non tollerabili.
Il problema della variazione della curva di risposta del sensore nel tempo si risolve calibrandolo ogni tanto – I sensori (tutti) hanno un drift, cioè la loro risposta varia nel tempo:
\begin{itemize}
\item Un ottimo, seppur raro, controllo di qualità consiste nel tenere ogni sensore sotto continua osservazione per un lungo periodo (4-6 mesi), durante il quale il sensore viene continuamente ricalibrato per verificarne la stabilità. Solamente così è possibile verificare l'effettiva stabilità nel tempo del comportamento di ogni singolo sensore;
\item I costruttori di sensori ad alta precisione, come Sea-Bird, seguono questa pratica, e scartano i sensori che non dimostrano una buona stabilità durante il periodo di valutazione – In funzione della tecnologia alla base del sensore e della sua costruzione la variazione può essere casuale o lineare e modellabile;
\item Variazioni casuali della risposta del sensore richiedono frequenti calibrazioni per avere misure affidabili;
\item Variazioni lineari e modellabili consentono calibrazioni meno frequenti – Tra calibrazioni successive si utilizzano termini interpolati;
\end{itemize}

\section{Acquisizione e al trattamento dei segnali}
\subsection{Il segnale}
Si definisce segnale la variazione di una qualsiasi grandezza fisica in funzione del tempo. 
In generale possiamo affermare che un segnale, sotto opportune ipotesi, può essere descritto matematicamente da una funzione continua. Le onde di volume viaggiano all'interno della terra e seguono le leggi dell'ottica geometrica. Vi sono due tipi principali di onde di volume:
\begin{enumerate}
\item Onde P: (prime) onde di compressione e rarefazione in cui l'oscillazione delle particelle di matteria avviene parallelamente alla direzione di propagazione dell'onda; 
\item Onde S: (seconde o di taglio) in cui l'oscillazione delle particelle avviene perpendicolarmente alla direzione di propagazione dell'onda.
\end{enumerate}
I Segnali Possono essere di 3 tipi:
\begin{enumerate}
\item Continui o Analogici: quando il loro valore può essere misurato in ogni istante. L'aggettivo 'analogico' è usato in relazione al fatto che la loro forma in uscita da un sistema è analoga a quella in ingresso.
\item Discreti: quando il loro valore può essere misurato solo in determinati istanti di tempo.
\item Digitali: quando si tratta di segnali discreti la cui ampiezza è associata ad un numero che ne rappresenta il valore in quell'istante. In genere per esprimere il numero viene usata, per semplicità, la base numerica binaria {0,1}.
\end{enumerate}
La teoria dei segnali si suddivide in due grandi branche a seconda del tipo di segnale in esame: 
\begin{enumerate}
\item Segnali deterministici: quando è possibile prevedere con certezza il suo valore ad un istante assegnato una volta conosciuti pochi parametri. Per esempio: una sinusoide di cui sia conosciuta l'ampiezza, la frequenza, e la fase ad un istante dato è un segnale deterministico.
Purtroppo i segnali deterministici non esistono nella realtà, in quanto ogni segnale fisico reale possiede una certa indeterminazione per cui il suo valore non può essere mai previsto con infinita esattezza: risultano però una buona approssimazione in molti casi pratici.
\item Segnali aleatori: quando è solo possibile prevedere solamente la probabilità che il segnale assuma un certo valore ad un dato istante. Un segnale aleatorio viene definito completamente da due funzioni: la densità di probabilità e la funzione di correlazione. La prima restituisce la probabilità che il segnale assuma un valore dato, la seconda fornisce una relazione tra i valori del segnale a istanti di tempo diversi.
\end{enumerate}
I casi più importanti dei segnali aleatori sono quelli in cui la distribuzione del segnale è gaussiana e quello in cui il valore del segnale ad un dato istante non è correlato a quello degli istanti precedenti e successivi. Esso prende il nome di rumore bianco.

\subsection{Lettura del segnale}
Per leggere un segnale bisogna utilizzare un trasduttore. Le caratteristiche importanti che un trasduttore deve possedere sono:
\begin{itemize}
\item l'ampiezza del segnale in uscita;
\item la sensibilità, ovvero il minimo valore misurabile;
\item la velocità di risposta, ovvero il tempo impiegato dal sensore per fornire una risposta costante dopo una brusca variazione della grandezza da misurare.
\end{itemize}
Da quest'ultima caratteristica dipende dalla banda passante, ovvero la massima frequenza di variazione del segnale a cui il sensore è in grado di funzionare. 
Un trasduttore è uno strumento di misura, pertanto la tensione fornita sarà affetta da una piccola componente casuale, che porterà la misurazione a fluttuare intorno al valore medio. Si schematizza questo fenomeno considerando il segnale prodotto dal trasduttore come la somma di una parte proporzionale al segnale vero e proprio che si intende misurare, e da un segnale aleatorio, il rumore bianco. Le cause delle fluttuazioni (o sorgenti di rumore) sono molteplici ad esempio:
\begin{itemize}
\item la temperatura (rumore termico);
\item la quantizzazione della carica dell'elettrone;
\item i campi elettrici presenti nelle vicinanze dell'apparecchio (la linea a 50 Hz);
\item le vibrazioni meccaniche;
\item \dots
\end{itemize}
Ulteriore rumore verrà introdotto quando il segnale prodotto dal trasduttore verrà trasportato lungo una linea elettrica, o amplificato, o comunque manipolato. Per permettere ad un calcolatore di elaborare un segnale, bisogna dapprima convertirlo in un segnale di tipo elettrico. In questo modo è possibile inviare il segnale ad una scheda di acquisizione che trasforma il segnale in una tabella di numeri binari interi, gli unici in grado di essere elaborati dal calcolatore, che poi l'utente leggerà come numeri decimali, dopo una appropriata conversione. Si rendono quindi necessarie due distinte operazioni sul segnale: 
\begin{itemize}
\item la discretizzazione: consiste nel misurare l'ampiezza del segnale ad intervalli di tempo fissati;
\item la quantizzazione: consiste nella trasformazione dei valori misurati in numeri interi binari.
\end{itemize} 

IMMAGINE DA CERCARE
Schema a blocchi di un convertitore A/D (analogico digitale)
Campionatore -> Quantizzatore -> Codificatore
Il processo di conversione A/D, che trasforma il segnale originario in una sequenza di bit {0,1}, è noto come tecnica PCM (Pulse Code Modulation o Modulazione impulsiva codificata). Gli A/D convertono i valori di tensione in ingresso nel numero corrispondente espresso in binario.
\subsection{Campionamento o discretizzazione}
La discretizzazione consiste nel misurare l'ampiezza del segnale ad intervalli di tempo fissati. Sia /Delta T l'intervallo di tempo tra due misure successive: la discretizzazione genera un vettore x di dimensione n. È evidente che l'intervallo di tempo /Delta T deve essere sufficientemente piccolo da riuscire ad individuare anche piccole variazioni de segnale. L'inverso dell'intervallo /Delta T si chiama frequenza di campionamento ed indica quante volte al secondo viene misurata l'ampiezza del segnale. 
\subsection{Quantizzazione}
La quantizzazione consiste nella trasformazione delle ampiezze misurate in numeri interi binari ad N bit. Supponiamo che la scheda di acquisizione accetti in ingresso una tensione positiva al massimo di $ V_{0} $ volt. Il numero di intervalli che si possono ottenere con N bit è dato da $2N$, il doppio. Allora il modo più efficiente di convertire il segnale è di dividere l'intervallo $ V_{0} $ in $2N$ sezioni, ciascuna di spessore $ V_{0}/2N $. Dopo di che bisogna assegnare ad ognuna di queste sezioni un numero binario ad N bit. Quando il segnale cade in una di queste sezioni, gli viene assegnato il numero binario corrispondente. Così facendo si commette in ogni misura al massimo un errore pari a metà dello spessore della sezione, $ V_{0}/2N+1 $. La minima variazione rilevabile del segnale in ingresso risulterà essere $ V_{0}/2N $. Ora si suppone che il trasduttore produca in uscita un segnale compreso tra un massimo $ V_{max} $ ed un minimo $ V_{min} $. Visto che la precisione con cui viene quantizzato il segnale dipende solamente dalle caratteristiche della scheda di acquisizione, per ottenere la massima risoluzione è necessario che $V_{max}$ e $V_{min}$ rientrino all'interno dell'intervallo permesso dalla scheda di acquisizione e che vi si adattino al meglio. 


A questo scopo è necessario amplificare o attenuare il segnale fornito dal trasduttore, ed eventualmente fornire una tensione aggiuntiva per evitare valori negativi. Queste operazioni costituiscono l'esempio più semplice del cosiddetto \textbf{condizionamento del segnale}, che può prevedere anche un filtraggio, la sottrazione di una componente continua, etc. Molte di queste operazioni vengono effettuate automaticamente dalla scheda di acquisizione stessa. Supponiamo di voler campionare una sinusoide di frequenza $f_{0}$. Intuitivamente, per riuscire ad ottenere un campionamento che rappresenti realisticamente il segnale dobbiamo essere in grado di misurare almeno due volte l'ampiezza del segnale all'interno di un periodo, in modo tale da riuscire ad evidenziare come il segnale oscilli e cambi di segno. Quindi la frequenza di campionamento minima per riuscire a ricostruire il segnale è pari al doppio della frequenza $f_{0}$. In altre parole, se il segnale viene campionato ad una frequenza $f_{s}$, allora sarà possibile campionare correttamente solo i segnali di frequenza minore di $f_{N}=f_{s}/2$. La frequenza $f_{N}$ così definita viene definita \textbf{frequenza di Nyquist}. Un segnale con una certa frequenza, se campionato con una frequenza leggermente superiore, appare con una frequenza molto più bassa di quella reale. Il fenomeno per cui le frequenze maggiori della frequenza di Nyquist vengono riconosciute come frequenze inferiori è detto \textbf{aliasing}.
Per evitare l'aliasing si utilizza un filtro passo basso, che elimina tutte le frequenze superiori a un certo valore (la metà della frequenza di campionamento). Il valore è detto frequenza di cut-off.
Sono necessari almeno due campioni per periodo del segnale:
\begin{enumerate}
\item La frequenza di campionamento deve essere almeno il doppio della max frequenza presente nel segnale;
\item La frequenza di Nyquist.
\end{enumerate}
La quantizzazione è un processo non-lineare, ed a differenza del processo di campionamento, non è reversibile, indi per cui non è possibile ricostruire i valori reali assunti originariamente dalla grandezza fisica. La quantizzazione è dunque una fonte di distorsione.
Una volta che il segnale è stato campionato, deve essere quantizzata anche l'ampiezza dei valori campionati. Un quantizzatore si occupa di associare ad un range di valori d'ingresso un unico valore di uscita. Questo deve avvenire in quanto, in un segnale analogico, il range di valori tra due estremi è infinito mentre nel mondo digitale solo un certo numero di valori potrà essere rappresentato. La quantizzazione può essere di tipo scalare (SQ) oppure vettoriale (VQ). Nella SQ, ciascun campione viene quantizzato singolarmente, mentre nella VQ vengono quantizzati congiuntamente blocchi di campioni. A sua volta i modi secondo cui il quantizzatore può operare sono diversi, i più comuni sono la quantizzazione uniforme e la quantizzazione a minimo errore quadratico medio.
\subsubsection{Uniforme}
Nel caso di quantizzazione uniforme, il gradino $\gamma$ di quantizzazione è uniforme cioè sempre della stessa ampiezza. Questo gradino (detto anche passo di quantizzazione) viene determinato a partire dalla massima dinamica del segnale di ingresso e dal numero di bit che abbiamo a disposizione. Definiamo come $R=[x_{max},x_{min}]$ il range dinamico del segnale di ingresso e come $n$ il numero di bit così da ottenere $\gamma=R/2n$. Ovviamente agendo in questo modo la quantizzazione introduce un errore, detto errore di quantizzazione. In correlazione all'errore di quantizzazione possiamo tenere in considerazione il rapporto tra segnale e rumore di quantizzazione detto SQNR. Esso rappresenta un indice di qualità relativo alla quantizzazione e l'errore di quantizzazione che, inevitabilmente si ottiene quantizzando il segnale campionato. E' possibile dimostrare che questo rapporto aumenta di 6 dB per ogni bit aggiuntivo a disposizione della quantizzazione. Questo è intuitivo perché aumentando il numero di bit a disposizione per la quantizzazione, il passo $\gamma$ diventa più piccolo ed il segnale analogico può venire quantizzato in maniera più precisa. Per la descrivere un segnale continuo abbiamo bisogno una sequenza più lunga di valori discreti, e quindi sarà maggiore è il numero di bit usati e maggiore sarà l'accuratezza della descrizione. Di conseguenza più gradini ci sono minore sarà l'errore di quantizzazione. 

\subsubsection{Non uniforme}
La quantizzazione non uniforme slega SQNR e $\gamma$ dinamica. Con non uniforme si indicano intervalli spaziati differentemente ovvero i gradini $\gamma$ non hanno la stessa dimensione. Questa tecnica è utilizzata quando la statistica del segnale in ingresso non è uniforme. Questo vuol dire che si presentano intervalli più piccoli, e quindi più accurati, dove la concentrazione di valori dei campioni è più alta rispetto a intervalli più grandi e quindi meno accurati, dove la concentrazione di valori di campioni è più bassa. Così facendo si presta più attenzione nella quantizzazione di valori che si presentano con maggiore probabilità diminuendo così l'errore di quantizzazione. Lo scopo della quantizzazione non lineare è quello di migliorare l'errore relativo ai bassi livelli del segnale attraverso la quantizzazione non lineare analogica oppure la quantizzazione non lineare digitale. \\
Nella quantizzazione non lineare analogica, il segnale PAM (pulse-amplitude modulation) prima di essere convertito in digitale attraversa un compressore analogico, sostanzialmente un amplificatore logaritmico, che ha lo scopo di amplificare i livelli più bassi del segnale PAM e comprimere quelli alti. L'apparato ricevente PCM deve contenere un organo denominato espansore, complementare al compressore in grado di ripristinare i livelli originali.\\
Nella quantizzazione non lineare digitale, invece, il segnale PAM è convertito immediatamente in digitale a 12 bit. Successivamente un compressore numerico trasforma i dati in forma seriale a 8 bit. Anche la compressione numerica segue una legge logaritmica simile a quella analogica. L'apparato ricevente PCM deve contenere un organo denominato espansore, complementare al compressore in grado di ripristinare i livelli originali. Un compressore digitale trasforma una parola a 12 bit (4096 combinazioni) in una a 8 bit (256 combinazioni). La curva del compressore analogico è approssimata con una spezzata costituita, secondo le norme CEPT (Conferenza Europea delle amministrazioni delle Poste e delle Telecomunicazioni), di 8 segmenti ricavati dividendo in 8 parti uguali l'asse delle ordinate. L'asse delle ascisse è così diviso in segmenti chiamati "segmenti di tratta". Dato in ingresso il codice a 12 bit:\\ 
$N_{in}$ = [S, N10, N9, N8, N7, N6, N5, N4, N3, N2, N1, N0] \\ 
Il compressore numerico fa corrispondere il codice a 8 bit: \\
$N_{out}$ = [S, A, B, C, D, E, F, G] \\
Dove S è il bit di segno (0 positivo e 1 negativo), [A, B, C] è il segmento di tratta e [D, E, F, G] è il codice binario corrispondente all'interpolazione lineare dentro la tratta.
La quantizzazione non-lineare è un processo irreversibile, che modifica il segnale originario, approssimandone il valore con uno vicino, ma non identico. Per questo il processo di quantizzazione introduce rumore. Il rumore di quantizzazione in uscita dal filtro di ricezione in un sistema di comunicazione è bianco e gaussiano. La non-linearità espande gli intervalli più vicini all'origine e comprime quelli verso il valore massimo. La cascata dei due blocchi emula un quantizzatore non lineare. Per migliorare le prestazioni in presenza di segnali che possono cambiare significativamente la dinamica bisogna aumentare il numero di bit del campione. Utilizzare quantizzatori non uniformi. Segnali con piccola dinamica vendono piccoli intervalli di quantizzazione, segnali con grande dinamica vedono intervalli di quatizzazione grandi. Il quantizzatore non lineare è ottenuto con un blocco non-lineare (compressore) posto a monte di un quantizzatore uniforme. Un blocco inverso al compressore (espansore) è utilizzato dal lato ricostruzione.
\subsubsection{Logaritmica}
Assegna i valori a regioni uniformi sulla scala logaritmica
Vantaggi: produce risparmi di memoria ed ha una migliore qualità audio a parità di SR.
Svantaggio: più complesso applicare le tecniche di elaborazione del segnale (la somma di due segnali corrisponde al prodotto, proprietà dei logaritmi).
Gli esempi più importanti di quantizzazione logaritmica sono rappresentati dai quantizzatori A-law e $\mu$-law, utilizzati rispettivamente in Europa e negli USA per la telefonia fissa (PCM, 8000 Hz di campionamento, 8 bit per campione).

\subsection{Codificatore o encoder}
Un codificatore o encoder è un circuito digitale combinatorio dotato di $2^{n}$ segnali di ingresso e di $n$ segnali di uscita. L'attivazione di una delle linee di ingresso produce in 
uscita il codice corrispondente. Si utilizza in combinazione con i trasduttori digitali che forniscono in ingresso del codificatore un segnale binario. Tra ingresso e uscita del codificatore non esiste però un legame logico, come caso del decoder, perché all'interno di esso esistono delle allocazioni perenni di memoria (memorizzate dal costruttore) tali che il loro numero sia pari alle linee in ingresso (ogni linea attiva individua una locazione di memoria). Se gli ingressi attivati sono più di uno, l'uscita potrebbe assumere una configurazione binaria indesiderata. Per evitare che questo accada, i codificatori in commercio sono possiedono una priorità: se si attiva più di una linea in ingresso, l'uscita assumerà la configurazione associata all'ingresso con priorità maggiore tra quelli attivati. 




\chapter{Cap2}
\section{Internet of Things}
L'Internet of Things tocca tutti gli aspetti della vita quotidiana, coprendo una gamma di applicazioni inimmaginabile, dai dispositivi indossabili connessi, progettati per comunicare su distanze di pochi centimetri, a un'ampia varietà di applicazioni di gestione delle risorse e di monitoraggio a sensori, che potranno comunicare con uno o più gateway su distanze anche di diversi km. Nel caso in cui siano necessarie comunicazioni a lungo raggio e non possa essere utilizzato un semplice data link wireless senza autorizzazione, le reti cellulari costituiscono un attraente mezzo di collegamento, anche se presentano alcuni inconvenienti. Anche se un dispositivo IoT può essere connesso a basso costo a una rete 2G, con un consumo abbastanza contenuto da poter funzionare con alimentazione a batteria per una durata accettabile, c'è comunque qualche incertezza sul futuro delle reti 2G. Alcuni operatori hanno espresso l'intenzione di abbandonare queste reti, dato che gli abbonati sono orientati ai più moderni servizi 3G e 4G, che garantiscono ai dispositivi mobili un miglior collegamento Internet. I dispositivi IoT si indirizzano di solito alla realizzazione di servizi di lunga durata, dai cinque agli otto anni. Occorre quindi scegliere una connettività di rete che sia sicuramente supportata per questo periodo. A causa dell'incertezza sulla longevità delle reti 2G, gli sviluppatori devono prendere in considerazione soluzioni alternative di connettività che siano in grado di garantire non solo la certezza di un supporto a lungo termine, ma anche il rispetto delle esigenze di basso consumo, comunicazioni a lungo raggio e basso costo, tipiche delle più diffuse applicazioni IoT. Fra i possibili candidati, la tecnologia LoRa. \cite{art:rif.1} 

Gli investimenti Venture Capital (VC) in questi ultimi anni hanno subito una decisa accelerazione nel settore IoT: a livello mondiale tra il 2010 e il 2015 sono più che raddoppiati passando da 0,8 miliardi di dollari nel 2010, a circa 2 miliardi di dollari nel 2015. A livello cumulativo, il montante degli investimenti in start up specializzate nell’IoT negli ultimi sei anni ha raggiunto la quota di 7,4 miliardi di dollari, per un totale di 887 operazioni. Tra i principali investitori figurano importanti attori del Corporate VC: questo significa che le divisioni di imprese “non finanziarie” hanno l’obiettivo di investire in start up con tecnologie utili al proprio business. Tra gli operatori più attivi sono presenti Cisco Investments, Intel Capital, Google Ventures, GE Ventures e Qualcomm Ventures il cui portafoglio di investimenti in start up specializzate nell’IoT è stato stimato, nel periodo che va dal 2010 al 2015, in circa 3,2 miliardi di dollari. Questi solo alcuni dei dati emersi dal Rapporto Speciale Looking Forward di Accenture, interamente dedicato al tema dell’Internet of Things.
Il percorso di trasformazione digitale in atto nelle imprese e nei mercati apre scenari economici e di business totalmente inediti, attraverso nuovi modelli di apprendimento, collaborazione, “contaminazione” fra imprese, a cavallo tra competizione e gestione. Questo processo interessa al tempo stesso sia le imprese che le istituzioni preposte alla guida dei mercati e dell’economia che devono essere in grado di porre nei tempi adeguati le condizioni infrastrutturali, regolamentari, di formazione per raggiungere in maniera efficace gli obiettivi di crescita diffusa e innovazione. Ad esempio nel 2020 oltre il 35\% della popolazione delle economie mature avrà almeno un dispositivo wearable (indossabile). I consumatori si interfacceranno con ecosistemi digitali e fisici non solo grazie a device tradizionali come PC, smartphone e tablet, ma tramite una moltitudine di oggetti e sensori intelligenti. Questo fenomeno si sposa con l’opportunità di fornire migliori servizi ai consumatori: già oggi infatti il 40\% dei consumatori si dichiara disponibile a condividere dati personali, a patto che questi vengano utilizzati per ricevere servizi disegnati quasi in modo sartoriale. L’applicazione dell’Internet of Things non avrà impatti diretti solo sul cliente finale, ma anche sui processi interni delle aziende volti all’interazione con i clienti stessi. Sensori e oggetti intelligenti posizionati in modo diffuso negli uffici e nei punti vendita consentiranno di ottimizzare i processi di gestione del cliente, favorendo la collaborazione e garantendo un migliore customer service. I grandi operatori della logistica come UPS e DHL stanno già sfruttando le reti di sensori per localizzare e gestire al meglio le loro flotte e per consentire la verifica dello stato di consegna da parte del cliente. Ad esempio in ambito B2B, i produttori di macchinari per movimento terra o attrezzature agricole potrebbero migliorare l’efficienza operativa monitorando in real time le macchine, consentendo anche operazioni di manutenzione predittiva e diminuendo i fermi macchina.
 
Non sono però solo le grandi multinazionali ad investire nel comparto dell’Internet of Things numerosi Paesi stanno acquisendo sempre più consapevolezza delle opportunità economiche del comparto, promuovendo numerose iniziative per dare al proprio ecosistema dell’innovazione un ruolo leader che accompagni la società in questa trasformazione tecnologica. E’ quanto sta accadendo in Francia, ad esempio, che ha inaugurato la “Cité de l’objet connecté” ovvero uno spazio di 10.000 metri quadrati messo a disposizione degli startupper per progettare e realizzare oggetti intelligenti e connessi; in Germania l’IoT è stato  classificato come uno tra i settori prioritari in cui concentrare gli investimenti relativi al piano Industry 4.0. La Cina ha investito 800 milioni di dollari per sviluppare il comparto dell’IoT e corposi investimenti vengono portati avanti anche da Corea del Sud e Usa.
\subsection{Italia e IoT} 
Anche il nostro Paese sta attivando diverse iniziative per dare una ulteriore spinta all’ecosistema dell’innovazione in generale e all’universo dell’IoT in particolare. Secondo la ricerca Accenture “Industrial Internet of Things” l’Italia è uno dei Paesi con le maggiori opportunità di crescita. Lo studio ha evidenziato che investimenti aggiuntivi in questo settore porterebbero a un incremento stimato di produttività  pari a 197 miliardi di dollari entro il 2030.
Inoltre la vocazione alla tradizione manifatturiera rappresenterebbe un terreno fertile per l’applicazione di tutte le tecnologie di automazione dei processi industriali, per abilitare la fabbrica del futuro (Fabbrica 4.0). Inoltre il posizionamento distintivo e di eccellenza su diversi settori che stanno attualmente guidando la volata all’IoT, come i settori auto, casa, automazione e salute, può essere ulteriore spinta e volano per la trasformazione digitale anche delle filiere legate a tali comparti.
Anche la Pubblica Amminisrazione sembra consapevole della centralità del tema facendo del recupero del gap d’innovazione una priorità strategica per l’Italia che viene confermata da alcuni recenti interventi normativi finalizzati a supportare anche il mercato IoT.\cite{art:rif.4}
\subsection{SigFox}
Esistono oggi opzioni sempre più interessanti per implementare collegamenti di tipo wireless nelle reti di sensori e in numerose altre applicazioni Internet of Things. Per un decennio o forse più, le reti di telefonia mobile (cellulari) hanno rappresentato l'unica tecnologia di comunicazione wireless universale disponibile per produttori e operatori di apparecchiature M2M in grado di garantire una copertura praticamente globale in ogni regione abitata del pianeta. Nel caso delle applicazioni M2M, si è deciso di utilizzare la tecnologia GPRS come base per la connessione alla rete di telefonia mobile, mentre le più recenti tecnologie 3G e 4G (di terza e quarta generazione) garantiscono velocità di trasferimento sempre maggiore a fronte di un aumento dei costi di connessione. Tutte queste tecnologie per telefoni mobili evidenziano svantaggi non indifferenti per gli utilizzatori di apparecchiature M2M: la velocità di trasmissione dati è molto più elevata rispetto a quella richiesta da un gran numero di applicazioni M2M per cui i moduli cellulari integrati nelle apparecchiature IoT sono sovra-specificati e quindi troppo costosi per tali applicazioni. Senza dimenticare che le elevate tariffe che gli operatori di reti mobili impongono per collegare anche il più semplice dei dispositivi wireless sono proporzionali alle elevate velocità di trasferimento dati che la rete può supportare. Un altro aspetto da tenere in considerazione è che le prestazioni offerte dalla tecnologia per telefoni mobili tendono a deteriorarsi in condizioni ambientali severe o estreme. In sintesi, nella maggioranza delle applicazioni M2M, l'utilizzo di una rete di telefonia mobile per garantire la copertura wireless universale risulta costoso. Gli utenti hanno ora la possibilità di scegliere tra due nuove tipologie di reti Wan (Wide Area Network)ciascuna delle quali garantisce sensibili risparmi in termini di costi rispetto alle reti di telefonia mobile. Scopo dell'articolo è confrontare queste nuove tecnologie per applicazioni M2M evidenziando pregi e difetti.
\subsubsection{Bassi consumi abbinati a una copertura geografica}
Entrambe le reti oggetto dell'articolo rientrano in una nuova categoria di rete universale denominata Lpwan (Low-Power Wide-Area Network) pubblica. Il nome, in realtà, suona un po' paradossale: gli approcci di tipo tradizionale alla connettività wireless suggeriscono che un dispositivo di rete non dovrebbe essere in grado di operare con consumi ridotti mentre contemporaneamente sta trasmettendo su lunghe distanze. La topologia dei due tipi di reti è la stessa di quella utilizzata dalle tecnologie per telefonia cellulare, ovvero del tipo a stella con una stazione Bts (Base Transceiver Station) al centro. A differenza dei sistemi 2G, 3G o 4G, una rete Lpwan adotta uno schema di modulazione che "penalizza" la velocità di trasmissione dati (throughput) al fine di garantire una maggiore tolleranza nei confronti delle interferenze e dell'attenuazione del segnale. In questo modo la potenza di trasmissione (in uscita) potrà essere molto bassa. Nello stesso tempo la tecnologia richiede ricevitori caratterizzati da una sensibilità molto elevata al fine di mantenere una connessione in presenza di segnali di ingresso relativamente deboli. In altre parole, a differenza di una rete di telefonia mobile, una rete Lpwan è ottimizzata per l'utilizzo in applicazioni M2M e IoT, che richiedono bassi consumi e ridotta velocità di trasferimento dati. Di conseguenza una cella Lpwan può garantire un'ampia copertura, potenzialmente persino superiore rispetto a quella di una cella di telefonia mobile, utilizzando una potenza inferiore.
\subsubsection{Due possibili tecnologie}
Entrambe le nuove tecnologie Lpwan operano a frequenze comprese nelle bande Ism esente da licenze. A differenza degli operatori di reti di telefonia mobile, quindi, gli operatori di reti Lpwan non devono acquistare costose licenze per l'assegnazione di bande dello spettro radio. Nonostante ciò, il costo per creare un'infrastruttura wireless pubblica è notevole e il tempo richiesto per raggiungere un grado di copertura tale da permettere alle nuove reti Lpwan pubbliche di soddisfare le esigenze di una vasta platea di utenti (e non solo di una nicchia) è considerevole. Ora, però, le due tecnologie Lpwan si propongono come una valida alternativa per Oem e utilizzatori. La prima tecnologia è SigFox mentre la seconda è LoRa, una tecnologia Lpwan sviluppata da Semtech, azienda produttrice di semiconduttori.
\subsubsection{Reti pubbliche SigFox}
La rete pubblica SigFox copre Francia, Spagna, Gran Bretagna e Paesi Bassi e a partire dal mese di luglio del 2015 sono state condotte numerose prove sul campo in numerose città in tutto il mondo, mentre l'installazione di una rete su scala nazionale è stata avviata in Portogallo, Danimarca, Belgio e Stati Uniti. L'obiettivo di SigFox prevede la copertura su scala nazionale in oltre 60 Paesi entro il 2020. In Francia, SigFox possiede e gestisce la rete, sviluppa l'ecosistema locale e vende abbonamenti ai servizi di comunicazione, mentre in altri Paesi queste attività sono demandate alla responsabilità di partner Sno (SigFox Network Operator). Gli operatori Sno forniscono un servizio di comunicazione end-to-end che trasferisce in modo sicuro i dati dai dispositivi remote agli application server degli utilizzatori. Esso offre un'interoperabilità nativa e il supporto per il roaming, elementi critici per garantire la connettività su scala globale dei dispositivi IoT. Grazie alla topologia tipica della tecnologia SigFox e alla propagazione del segnale su lunghe distanze, l'investimento richiesto per implementare una rete SigFox è molto minore rispetto a quello necessario per i sistemi cellulari tradizionali, consentendo in tal modo agli operatori SNO di fornire la connettività agli utenti Oem a prezzi accessibili. Un Oem che desidera abbonarsi alla rete pubblica SigFox deve disporre di un modulo client che fa girare lo stack client SigFox e un transceiver radio operante a 868MHz in grado di effettuare la modulazione Dbpsk(Differential Binary Phase-Shift Keying) per l'uplink la modulazione Gfsk (Gaussian Frequency Shift Keying) per il downlink. Alcuni Oem potrebbero decidere di sviluppare in proprio il progetto o il modulo mentre molti altri opteranno per un modulo Cots già pronto all'uso certificato SigFox Ready. I gateway e tutto il software per il collegamento in rete e applicativo per il trasporto dei dati sono forniti da SigFox al fine di assicurare la medesima qualità di fruizione indipendentemente dal Paese in cui gli oggetti stanno comunicando. Poiché secondo SigFox la distanza di trasmissione in campo aperto può essere superiore a 15 km, è possibile creare una rete con copertura universale utilizzando un numero relativamente ridotto di celle. Più stazioni base possono ricevere e trasmettere lo stesso messaggio: questa diversità in ricezione nativa abbinata alla reiezione dell'interferenza dei segnali a banda ultra-stretta (UNB – Ultra Narrow Band)contribuisce a garantire un'elevata affidabilità della trasmissione. La ridotta spaziatura tra i canali nella trasmissione in uplink (ovvero verso la rete) implica un'elevata selettività della stazione base. Questa è resa possibile dall'adozione di ricevitori Sdr (Software-Defined Radio) cognitivi (che cioè effettuano una scansione dinamica dello spettro radio). In questo modo è possibile ridurre la complessità del modem del prodotto finale, che per l'Oem si traduce in una riduzione dei costi di implementazione. SigFox non utilizza uno schema di modulazione proprietario per cui produttori di semiconduttori e costruttori di moduli possono realizzare trasmettitori e transceiver conformi alle specifiche SigFox. Un esempio è rappresentato dalla famiglia di prodotti SigFox compatibili della famiglia ATA8520x di Atmel, la quale ha anche annunciato l'introduzione di un transceiver RF SigFox completamente integrato. I moduli SigFox di tipo Cots sono caratterizzati da una sensibilità massima pari a -126dBm. La potenza di uscita irradiata per la banda Ism è stata stabilita da Etsi ed è pari a 14dBm.
\subsubsection{SigFox: prestazioni, costi e limitazioni}
In un sistema SigFox il numero di trasmissioni giornaliero è limitato a 140 messaggi in uplink, ciascuno composto da un massimo di 12 byte, e a soli 4 messaggi in downlink, composti da un massimo di 8 byte. La latenza è dell'ordine di 3-5 ms. Sulla base di queste caratteristiche, SigFox è adatta all'uso in applicazioni che prevedono una trasmissione occasionale di piccoli pacchetti di dati, in cui quindi il sistema resta per lunghi periodi in uno stato di inattività (power-down) al fine di preservare la durata della batteria. I contatori per la contabilizzazione dei consumi sono un ottimo esempio di un tipico esempio di una rete SigFox. Gli utenti di SigFox devono pagare una quota annuale per ciascun nodo per la fornitura dei servizi di comunicazione di rete. Questa quota annuale è determinata dagli operatori di rete in ogni Paese ma SigFox sottolinea che in ogni caso gli utenti possono beneficiare di prezzi estremamente competitivi abbinati a consumi di potenza minimi. Il percorso che ha portato allo sviluppo della rete LoRA (abbreviazione di Long Range) universale è differente da quello seguito da SigFox.
\cite{art:rif.5}
\section{LoRa}
\subsection{Che cos'è Lora?}
LoRa è la piattaforma wireless a lungo raggio e bassa potenza è la scelta tecnologica prevalente per la creazione di reti IoT in tutto il mondo. Le applicazioni Smart IoT hanno migliorato il nostro modo di interagire ed affrontare alcune delle più grandi sfide che le città e le comunità devono affrontare: cambiamenti climatici, controllo dell'inquinamento, allerta preventiva dei disastri naturali e salvataggio di vite umane. Anche le imprese ne beneficiano, grazie a miglioramenti nelle varie operazioni e alla loro efficacia, nonché alla riduzione dei costi. Questa tecnologia RF wireless viene integrata in automobili, lampioni stradali, apparecchiature di produzione, elettrodomestici, dispositivi indossabili, ecc... Insomma in qualsiasi cosa nel mondo reale, la tecnologia LoRa sta rendendo il nostro mondo un pianeta Smart, intelligente. L'IoT connette il nostro mondo. LoRa è il meccanismo che lo rende intelligente connettendo praticamente tutte le cose - sensori, gateway, macchine, dispositivi, animali, persone - La tecnologia LoRa consente di connettersi al cloud, consentendo decisioni corrette e migliorando la vita delle persone. La tecnologia LoRa di Semtech è una piattaforma per le tecnologie LPWAN (Low-Power Wide Network), progettata per inviare una piccola quantità di dati su lunghe distanze. È particolarmente adatto ai sensori e alle applicazioni di monitoraggio del clima che inviano i dati alcune volte all'ora, come i sistemi di allarme rapido, il controllo dell'inquinamento e le applicazioni di monitoraggio dei cambiamenti climatici. Keysight Technologies ha scelto la tecnologia LoRa di Semtech grazie al suo posizionamento nel mercato dei sensori IoT.LoRa presenta notevoli credenziali tecniche ed è già in uso in applicazioni che richiedono un'affidabile capacità di comunicazione su distanze di diversi km, come i sistemi wireless di lettura di strumenti e controllo dell'illuminazione stradale. 

\subsection{Semtech}
Nel campo dell'innovazione l'azienda Semtech con LoRa Technology offre un mix di strumenti molto interessante di lunga portata a basso consumo energetico e trasmissione sicura dei dati. Le reti pubbliche e private che utilizzano questa tecnologia possono fornire una copertura più ampia rispetto a quella delle reti cellulari esistenti. È più facile collegarsi ad un'infrastruttura esistente e che offre una soluzione per servizi di applicazioni IoT a batteria. Semtech sviluppa la tecnologia LoRa nei suoi chipset. Questi chipset sono quindi integrati nei prodotti offerti dalla vasta rete di partner IoT e integrati in LPWAN da operatori di reti mobili in tutto il mondo.
Semtech fornisce il silicio che incorpora la tecnologia RF wireless LoRa. I transistor sono ottimizzati esclusivamente per le comunicazioni IoT, costruiscono la connessione tra endpoint remoti, picocelli e gateway, quindi trasmettono tali informazioni al cloud per la consegna a cellulari e sistemi, rendendo il nostro mondo uno Smart Planet. I transceiver LoRa (trasmettitori) di Semtech RF sono incorporati nei sensori.
In oltre 60 paesi, Semtech ha collaborato con reti, pubbliche e private, e provider di servizi mobili per l'implementazione di reti a bassa potenza (LPWAN) basate sulla tecnologia LoRa di Semtech. Si ritiene che la chiave per il dimensionamento di Internet of Things sia un'infrastruttura di rete aperta e interoperabile, in modo che le applicazioni IoT sviluppate in un paese possano essere implementate in tutti. 
Semtech offre soluzioni di connettività wireless integrate, lunghe e corte. I prodotti RF wireless sono costituiti da transceiver RF, trasmettitore RF e componenti ricevitore RF che coprono lo spettro di frequenze radio in banda industriale, scientifica e medica (ISM) da basso kHz a 2,4 GHz. I nostri circuiti integrati RF wireless sono utilizzati in tutto il mondo per il controllo remoto wireless automatizzato, lettori di contatori, sistemi di sicurezza wireless, apparecchiature per l'automazione degli edifici e molto altro. Inoltre, i nuovi prodotti wireless LoRa® di lunga durata di Semtech sono la soluzione definitiva per eliminare i ripetitori, ridurre i costi, prolungare la durata della batteria e migliorare la capacità della rete. Semtech offre piattaforme di trasmissione e ricezione di potenza wireless per applicazioni di carica diretta e indiretta in entrambi i sistemi conformi agli standard e non conformi. 

\subsection{IBM Reserch}
IBM Research (NYSE:IBM) ha annunciato una nuova tecnologia basata su wide-area network a basso consumo(LPWAN), che offre vantaggi significativi alle reti cellulari e wifi per le comunicazioni machine to machine (M2M). Per anni, l'enorme potenziale dell’Internet of Things (IoT) per le aziende - raccolta dati da diversi dispositivi, la loro analisi ed elaborazione per un processo decisionale più rapido e veloce - è stato frenato da difficoltà tecniche quali una durata limitata della batteria, distanze di comunicazione brevi, costi elevati e mancanza di regole. La tecnologia chiamata LoRaWAN™ (Long Range wide-area network), risolve questi problemi. Basata su nuova specifica e protocollo per le reti wide-area a basso consumo che utilizzano uno spettro wireless senza licenza, la tecnologia è in grado di collegare i sensori sulle lunghe distanze, offrendo nel contempo una durata ottimale della batteria e richiedendo un'infrastruttura minima. Questo consente di offrire vantaggi quali mobilità, sicurezza, bi-direzionalità e localizzazione/posizionamento migliorati, oltre a costi più bassi.
Con i suoi 70 anni di attività, IBM Research continua a definire il futuro dell'informatica con più di 3.000 ricercatori in 12 laboratori situati in sei continenti. Gli scienziati di IBM Research hanno prodotto 6 Premi Nobel, 10 U.S. National Medals of Technology, 5 U.S. National Medals of Science, 6 Turing Awards, 19 membri della National Academy of Sciences e 14 membri dell'U.S. National Inventors Hall of Fame.

\subsection{Senet}
Senet Inc. è un operatore M2M Network as a Service (NaaS) con sede nel New Hampshire, sta attualmente installando 20.000 sensori Semtech LoRa con il software LRSC di IBM per monitorare i livelli del propano combustibile e delle taniche di petrolio situate presso edifici residenziali e aziende sulla east e la west coast degli Stati Uniti. Ogni ora i sensori raccolgono e trasmettono dati in modo sicuro, tra cui i livelli di carburante, lo stato degli indicatori, dei sensori e i rapporti di ricalibrazione dei sensori ai fornitori di carburante che stabiliscono quando effettuare le consegne e reintegrare le scorte. Gli indicatori di Senet sono precisi e, grazie alla tecnologia LoRa funzionano sulle lunghe distanze, riducendo i nostri costi infrastrutturali e consentendoci di ribaltare questi risparmi sui nostri clienti. I consumatori traggono inoltre vantaggio dal fatto di sapere quanto carburante è presente nel serbatoio in qualsiasi momento.
Senet, Inc. è leader nel mercato in rapida crescita dell'Internet of Things (IoT)/Machine-to-Machine (M2M) e come primo fornitore pubblico di Network as a Service (NaaS) negli USA per una rete ISM a basso costo e lunga distanza. Attraverso la sua rete presente in tutto il paese, Senet attiva servizi di monitoraggio che aiutano le aziende americane più efficienti ed attente all'ambiente a migliorare i profitti gestendo e misurando lo stato di beni diffusi -- attività come l'automazione dei serbatoi di propano e gasolio residenziale, misurazione di acqua e gas, distribuzione di lubrificanti commerciali, irradiazione solare e molto altro ancora.

\subsection{FastNet}
FastNet è fornitore leader di servizi di comunicazione di dati wireless. Con 20 anni di esperienza nell'introduzione delle comunicazioni Point-Of-Sale (POS) in Sud Africa, FastNet fornisce una rete conforme alla Payment Card Industry (PCI) affidabile, sicura, e soluzioni di comunicazione di dati end-to-dend ad aziende di qualsiasi dimensione. La società è specializzata in POS, Virtual Private Network, comunicazioni machine-to-machine e tecnologia Wi-Fi.
FastNet è ben posizionata per erogare un servizio di livello superiore e un'assistenza tecnica 24 ore su 24, 7 giorni su 7 in tutto il Sud Africa. FastNet è inoltre fornitore di servizi con licenza Electronic Communications Network Services (ECNS) e Electronic Communication Services (ECS) con il vantaggio di un'ampia copertura fornita da reti wireless e fisse.
FastNet è una società di proprietà al 100\% di Telkom SA SOC Limited. 

\subsubsection{LoRa Alliance}
%La diffusione sul mercato delle reti LoRA è supportato da LoRA Alliance, un'associazione fondata nel dicembre del 2014 che raggruppa numerosi produttori moduli per nodi terminali tra cui Semtech, IMST, Microchip, Multi Tech, Link Labs ed Embit; produttori di concentratori che utilizzano il chipset SX1301 tra cui IMST, Kerlink, MultiTech ed Embit; vari operatori di infrastrutture di rete; Ibm e Actility, fornitori di server cloud su cui gira il software LoRa_Wan_Server.%
A supporto della tecnologia LPWAN, IBM, Semtech e altre società hanno co-fondato \textbf{LoRa Alliance}, una coalizione di aziende impegnata a standardizzare il protocollo LoRaWAN che consente questa interoperabilità. La LoRa Alliance è un’associazione aperta, non-profit ha come mission lo sviluppo e la standardizzazione delle Low Power Wide Area Networks (LPWAN) implementate in tutto il mondo per l’attivazione di Internet of Things (IoT).  LoRa Alliance ha lo scopo di unire hardware e software basati sullo standard LoRaWAN per gli operatori delle telecomunicazioni e gli operatori di rete consentendo loro di offrire servizi IoT ad aziende e consumatori. Dai sensori alle macchine, ai monitor fino ai computer indossabili, collegare miliardi di dispositivi a breve potrebbe essere tanto semplice quanto inviare un SMS al provider di telefonia locale.
LoRa™ Alliance è un'associazione aperta, non-profit i cui membri credono che l'era dell'internet of things sia adesso. La sua mission è la standardizzazione delle Low Power Wide Area Networks (LPWAN) implementate in tutto il mondo per l'attivazione di Internet of Things (IoT), machine-to-machine (M2M) e della città intelligente oltre ad applicazioni industriali.

LoRa Alliance ha sviluppato uno stack di protocolli per reti di grandi dimensioni basato sulla propria tecnologia denominato LoRa Wan. Esso è formato da un client, un server e un firmware per l'inoltro dei pacchetti (packet forwarding). La disponibilità di LoRa Wan semplificherà l'introduzione di un gran numero di reti LoRA pubbliche e private di grandi dimensioni in un prossimo futuro. Grazie a LoRA Alliance e alla disponibilità dello stack di protocolli LoRa Wan gli operatori di rete, compresi gli attuali operatori di reti mobile, possono disporre di un ecosistema che permette loro di ridurre i costi e accelerare l'installazione e la messa in esercizio (deployment) di reti pubbliche LoRA. Grazie alla disponibilità di moduli concentratori proposti da costruttori come Kerlink, Embit, Imst e MultiTech è possibile sviluppare in tempi rapidi l'hardware per la realizzazione di stazioni radio base che supportano la tecnologia LoRA. I concentratori forniti da Future Electronics sono corredati con il software Ibm o Actility giù pre-caricato. Nel caso di utenti che desiderino collegare dispositivi su una rete privata, piuttosto che sfruttare una rete LoRA pubblica con la copertura richiesta, il software disponibile a bordo permette di accelerare e semplificare l'implementazione della rete stessa. A questo punto vale la pena segnalare che la possibilità di realizzare reti private non è prevista per gli utilizzatori della tecnologia SigFox.
Grazie al range di trasmissione di 15 km in campo aperto tra un concentratore e un nodo previsto dalla tecnologia LoRa è possibile creare celle di grandi dimensioni che permettano di ottenere rapidamente la copertura su una vasta area. Tutte le comunicazioni su una rete LoRA sono sicure grazie all'utilizzo della cifratura Aes con chiave a 128 bit. Lo stack dei protocolli LoRa Wan gestisce la velocità di trasferimento dati e la potenza di uscita in maniera adattativa al fine di ottimizzare sia i consumi di potenza sia l'intensità del segnale. Ciò significa che una rete pubblica implementata con lo stack LoRA Wan è in grado di offrire agli utilizzatori tutti i benefici (consumi ridotti, basso costo ed elevate sicurezza) intrinseci della tecnologia LoRA. Sono già numerosi i fornitori di servizi che hanno adottato questa tecnologia. Orange, uno dei principali operatori di reti mobili su scala mondiale, ha deciso di utilizzare la tecnologia LoRA per le reti Lpwan installate in Francia nel primo trimestre del 2016. Orange utilizzerà la rete per applicazioni smart city in numerose città francesi.

\subsubsection{Wireless Power Consortium (WPC)}
Semtech è membro del Wireless Power Consortium (WPC) ed è attiva nell'aiutare a definire i futuri standard per l'alimentazione wireless.


\subsection{Architettura di rete}
In termini di architettura di rete LoRa, i nodi sono disposti secondo una topologia che viene tipicamente definita "star-of-stars" con i gateway in modalità di bridge trasparente che trasmettono messaggi dai nodi finali fino al server centrale nel backend della rete.
I gateway sono collegati al server di rete tramite connessioni IP standard, mentre i dispositivi terminali utilizzano la comunicazione wireless a singolo hop su uno o più gateway. 
La comunicazione con i nodi finali è generalmente bi-direzionale ed è anche possibile la trasmissione in multicast (invio simultaneo a più nodi), funzionalità utile in caso di aggiornamenti o invio di messaggi in blocco al sistema. \cite{art:rif.3}
Per gli utenti che operano in campo industriale e utilizzano reti private chiuse Semtech ha introdotto transceiver per reti LoRA in grado di supportare un range di trasmissione di 15 km tra un nodo e un gateway. Prestazioni come quelle appena descritte si possono ottenere utilizzando i transceiver RF SX1272 di Semtech, operante nel range di frequenza di 860-1,020 MHz, e SX1276 che opera in un range più ampio, compreso tra 137 e 1.020 MHz. Nel transceiver SX1276 la sensibilità è pari a -148 dBm (valore di picco). Pubblica o privata che sia, una rete LoRa richiede la presenza di un concentratore posto al centro della topologia a stella mentre la comunicazione è bi-direzionale alternata (half duplex) in modo nativo. Il numero di nodi collegati a un concentratore dipende dall'applicazione e, più precisamente, dal numero di pacchetti che devono essere trasmessi in un determinato periodo di tempo. Per le applicazioni caratterizzate da un elevato numero di nodi terminali, Semtech ha sviluppato una soluzione ad hoc per il concentratore che prevede l'uso del chipset in banda base ad alta efficienza SX1301 e da due modulatori I/Q SX1257 I/Q. Un concentratore realizzato a partire da questi chip è in grado di gestire fino a 10.000 nodi.


%% Symphony Link ™ 

\subsection{Specifiche LoRaWAN}
\subsubsection{LoRaWAN 1.1 Specifiche}
LoRaWAN 1.1 è l'ultima versione di LoRaWAN ed offre le seguenti funzionalità:
\begin{itemize}
\item Supporto per il roaming di handover: consente il trasferimento del dispositivo finale da una rete LoRaWAN a un'altra. 
\item Roaming passivo: consente il trasferimento del dispositivo finale da una rete LoRaWAN con versioni precedenti. Il tutto è trasparente per il dispositivo finale.
I dispositivi finali bidirezionali con slot di ricezione programmati (Classe B) fanno parte dei miglioramenti delle specifiche e sono ora ufficialmente supportati.
\item Miglioramenti di sicurezza.
\end{itemize}
Al fine di supportare implementazioni eterogenee e non a aggiornamenti coordinati a livello globale, sia i dispositivi finali che le reti LoRaWAN 1.1 supporteranno la compatibilità con le versioni precedenti per interoperare con i loro pari LoRaWAN 1.0.x.

\subsubsection{LoRaWAN Backend Interfaces 1.0 Specifica}
LoRaWAN Backend Interfaces 1.0 abilita le seguenti funzionalità:
\begin{itemize}
\item Unisce il server di rete (NS), il server (JS) e il server delle applicazioni (AS).
\item Abilita il roaming sia per LoRaWAN 1.0.x (solo per il roaming passivo) e per le reti LoRaWAN 1.1 (sia in roaming passivo che in handover).
\item Identifica l'entità che memorizza le credenziali del dispositivo finale (comprese le chiavi di root) come JS. Può essere separato dalle reti e amministrato da un'entità indipendente dalle reti che il dispositivo finale potrebbe utilizzare. Ciò consente alle reti di scaricare la procedura di autenticazione su un sistema dedicato. Alla fine potrebbe essere collegato a questa terza parte.
\end{itemize}

\subsubsection{LoRaWAN 1.1 Parametri regionali rev. A}
I parametri regionali LoRaWAN 1.1 rev. Parametri radio specifici per regione per dispositivi finali LoRaWAN 1.1.

\subsection{Protocollo LoRa / Lv 1}
LoRa è un abbreviare di "Long Range", che riassume anche il vantaggio chiave della tecnologia wireless. LoRa è una modulazione radio a spettro esteso brevettata (EP2763321 dal 2013 e US7791415 del 2008) sviluppata da Cycleo (Grenoble, Francia) e acquisita da Semtech NASDAQ: SMTC nel 2012. LoRa è basata sulla tecnica CSS (Chirp Spread Spectrum), che la rende in grado di variare la lunghezza del cosiddetto fattore di spreading (compreso tra 6 e 12 bit) e l'ampiezza di banda in funzione della bit rate (ovvero il numero di bit trasmessi al secondo) richiesta nel range compreso tra 20bit/s a 41kbit/s. Grazie alla straordinaria modulazione del chirp, il collegamento wireless può raggiungere una sensibilità fino a -137 dBm e fino a 157 dB di budget di collegamento. Il trade-off è la velocità dati raggiungibile, che è nell'intervallo di kilobit al secondo. Questo definisce l'applicabilità della tecnologia - mentre non è adatta allo streaming video, è ben adatta per servire l'Internet of Things (IoT) e le applicazioni M2M. LoRa può essere utilizzato su un'ampia gamma di frequenze da 137 MHz a 1020 MHz. Questo include numerose bande ISM prive di licenza come 169 MHz, 433 MHz, 868 MHz e 915 MHz. Questo è un fattore chiave per implementazioni e interoperabilità a livello mondiale. Tecnologia realmente innovativa che fissa un nuovo punto di riferimento in termini di distanza di trasmissione e di consumi di potenza, la tecnologia LoRa adotta uno schema di modulazione digitale completamente asincrono. A differenza di SigFox, la tecnologia LoRa può essere utilizzata in reti sia private sia pubbliche. Le elevate prestazioni che la tecnologia LoRA è in grado di offrire sono testimoniate dalla sua capacità di ricevere segnali fino a -22dB al di sotto della soglia del rumore di fondo, abbinata alla reiezioni dei canali adiacenti di almeno 69dB con un offset di 25kHz, un valore migliore di 30dB rispetto a quello che si ottiene utilizzando la modulazione FSK a 868MHz sui medesimi transceiver. Un tempo le radio in banda Ism destinate ad applicazioni industriali e operanti a frequenze inferiori a 1GHz erano caratterizzate da un range in campo aperto limitato a 2 km. LoRa è lo strato fisico della pila ISO/OSI. LoRa implementa la modulazione wireless utilizzata per creare il collegamento di comunicazione a lungo raggio. Il suo schema di modulazione a largo spettro consente un'operatività a lungo raggio e un'elevata capacità di rete, con bassa potenza a radiofrequenza. Grazie alla richiesta energetica contenuta, l'end-point di una rete LoRa con alimentazione a batteria può funzionare per molti anni, una durata sufficiente in molte applicazioni, che ha un sensibile effetto sui costi operativi di una rete con numerosi end-point. Per favorire l'inizio degli sviluppi basati sulla tecnologia LoRa, esiste un portafoglio di moduli radio conformi alle specifiche LoRaWan per consentire di semplificare e accelerare l'integrazione di questa tecnologia nei dispositivi IoT. I moduli Microchip, come l'RN2483, sono dispositivi plug-and-play che integrano un completo sottosistema radio con un microcontrollore, le Eeprom di identità, il front-end a radiofrequenza e i relativi circuiti, oltre a un cristallo. Questi moduli sono fra i primi che hanno passato il collaudo LoRa Alliance Certification. La soluzione LoRa può essere etichettata come CDMA. Sta usando diversi fattori di diffusione (velocità dati chirp) e velocità di codifica per segnali multiplex su una singola frequenza. Ciò non solo aumenta la capacità della rete, ma consente anche l'adattamento dinamico delle velocità dei dati del dispositivo. I dispositivi con un migliore collegamento a un gateway (a causa della vicinanza, di un ambiente a bassa rumorosità, di una visuale non ostruita) possono utilizzare velocità di trasmissione dati superiori (fino a 11 kbps) e risparmiare batteria. I dispositivi con una scarsa qualità dei collegamenti possono aumentare il budget dei collegamenti utilizzando velocità di trasmissione inferiori, estendendo il raggio di collegamento effettivo a oltre 30 km in linea di vista.


\subsection{Protocollo LPWAN / Lv 2}
LoRaWAN (Low-power Wide area network) è una specifica del protocollo costruita sulla tecnologia LoRa sviluppata dalla LoRa Alliance. Utilizza lo spettro radio senza licenza nelle bande Industrial, Scientific e Medical (ISM) per consentire la comunicazione a bassa potenza ed ampia area tra i sensori remoti e i gateway collegati alla rete. LoRaWAN è il protocollo MAC per la rete di nodi LoRa ad alta capacità. È uno standard LPWAN aperto gestito dalla LoRa Alliance. Sfrutta le funzionalità LoRa appena descritte ed ottimizza la durata della batteria e la qualità del servizio per i nodi LoRa. Questo approccio basato su standard per la creazione di una LPWAN consente di configurare rapidamente reti IoT pubbliche o private ovunque utilizzando hardware e software che sia bidirezionale, sicuro, interoperabile e mobile, fornisca una localizzazione accurata e funzioni come previsto. LpWan su LoRa(tecnologia sub-GHz a basso consumo) supporta una velocità dei dati da 0,3 kbps a 50 kbps, in funzione della distanza e della durata dei messaggi. La distanza di trasmissione può raggiungere 15 - 20 km. Anche in un ambiente urbano ad alta densità si possono coprire distanze di comunicazione di oltre 2 km. 
Il protocollo è completamente bidirezionale, che consente una consegna affidabile dei messaggi (conferme). Comprende la definizione della crittografia end-to-end per la sicurezza e la riservatezza dei dati, la registrazione over-the-air dei nodi finali e la funzionalità multicast. Grazie al modello di antenna distribuita e ai gateway GPS abilitati, la rete è in grado di localizzare la posizione dei nodi, anche quando sono mobili.


I sensori LoRaWAN sono in grado di comunicare a distanze superiori ai 100 km (62 miglia) in ambienti favorevoli, 15 km (9 miglia) in ambienti semi-rurali e a più di 2 km (1,2 miglia) in ambienti urbani densamente popolati ad una velocità di dati da 300 bit a 100 kbit. Questo li rende adatti all'invio di quantità di dati contenute, come le coordinate GPS e le letture del clima che la banda larga non è in grado di raggiungere. I sensori richiedono inoltre pochissima energia, la maggior parte di loro può funzionare per più di 10 anni con una sola batteria AA e, inoltre, le chiavi AES128 rendono praticamente impossibile l'intercettazione e la manomissione delle comunicazioni.

I nodi finali di una rete LoRa devono adattarsi a diverse esigenze per soddisfare i requisiti molto diversi di quasi ogni tipo di applicazione IoT, il protocollo LoRaWAN supporta tre classi di terminali, suddivisi secondo le seguenti classi:
\subsubsection{Classe A – finali bidirezionali}
I nodi finali Classe A LoRaWAN permettono la comunicazione bidirezionale  in cui la trasmissione del collegamento verso monte di ciascun dispositivo finale è seguita da due, brevi finestre di ricezione downlink. Ogni trasmissione dal nodo alla rete è seguita da due brevi finestre di ricezione in ingresso al nodo. Lo slot di trasmissione pianificato dal dispositivo finale si basa sulle proprie esigenze di comunicazione con una piccola variazione basata su un tempo casuale (tipo di protocollo ALOHA). Questa operazione di classe A è il più basso sistema di dispositivo di alimentazione finale per applicazioni che richiedono solo comunicazioni di downlink dal server poco dopo che il dispositivo finale ha inviato una trasmissione di uplink. Le comunicazioni in downlink dal server vengono accodate automaticamente fino al successivo uplink pianificato. Qualsiasi altra comunicazione dal server al nodo avverrà al successivo scambio programmato.
\subsubsection{Classe B – finali bidirezionali con finestre di ricezione programmate}
I dispositivi di Classe B LoRa forniscono le funzionalità dei classe A con l’aggiunta della possibilità di programmare intervalli di ricezione di informazioni in orari prestabiliti. 
Per consentire al dispositivo finale di aprire la finestra di ricezione all'ora pianificata, riceve un segnale di sincronizzazione sincronizzata dal gateway. Ciò consente al server di sapere quando il dispositivo terminale è in ascolto. La sincronizzazione temporale della rete avviene tramite l’invio di segnali regolari nel tempo dal gateway verso il nodo.
\subsubsection{Classe C – finali bidirezionali con massimo numero di slot in ingresso}
I sensori Classe C LoRa permettono di avere il massimo livello di trasmissione server-nodo, ovvero, hanno finestre di ricezione quasi continuamente aperte, chiuse solo durante la trasmissione. La ricezione viene inibita solo nel momento in cui il nodo sta inviando informazioni verso la rete. Questa caratteristica rende i finali di Classe C particolarmente adatti alle reti LoRa caratterizzate da flussi di dati server-nodo superiori a quelli nodo-server.

\subsection{Protocollo IP / Lv 3} 
Rispetto agli altri standard, una rete LoRa è IP-based compatibile con IPv6, una caratteristica essenziale per ogni sviluppo di nuovi progetti IoT. Una rete LoRa comprende dei gateway per la connessione al server centrale di rete. Gli end-point comunicano con una topologia di rete a stella mediante un collegamento wireless single-hop ai gateway con la possibilità di collegarsi a più gateway, per garantire la ridondanza del collegamento. Per coprire una grande area è sufficiente un'infrastruttura leggera. Microchip ha effettuato la dimostrazione di una rete privata LoRa che copre la maggior parte della città di Monaco usando solo 7 gateway. \cite{art:rif.2} 
Scendendo nel dettaglio l'architettura di una rete LoRa prevede una tipologia a stella di stelle in cui il Gateway è un bridge trasparente per i messaggi tra Devices e il Network Server. I Gateway sono connessi al Network server tramite una connessione basata sollo standard IP, mentre i Device utilizzando una comunicazione wireless single-hop verso uno o più Gateway. La comunicazione verso i Device è in generale bidirezionale, ma può anche supportare il muticast per gestire l'aggiornamento o la distribuzione massiva di messaggi per ridurre i tempi di comunicazione.
Il cuore di ogni concentratore LoRa è un demodulatore LoRa multi-canale in grado di decodificare tutte le varianti di modulazione LoRa su più frequenze in parallelo. Un demodulatore LoRa standard per dispositivo terminale (LoRa modem come SX1276 o SX1272) può decodificare un solo tipo di modulazione su una frequenza. Attualmente, il demodulatore LoRa ampiamente utilizzato è SX1301 di Semtech. Questo chip è integrato in tutti i gateway LoRa nel nostro attuale portafoglio. A causa del crescente numero di produttori di gateway, sta diventando difficile individuare le differenze tra i gateway. Per un operatore di rete su larga scala, i principali fattori di distinzione dovrebbero essere le prestazioni radio (sensibilità, potenza di invio), la connessione del chip SX1301 all'MCU gateway e il supporto e la distribuzione del segnale PPS. SX1301 collegato su un bridge da USB a SPI ha una latenza più lunga sull'MCU rispetto a un SX1301 direttamente collegato al bus SPI dell'MCU. Attualmente, l'unico SPI collegato SX1301 si trova nella stazione IoT Kerlink, tutti gli altri gateway utilizzano ponti USB-to-SPI FTDI (o simili). La disponibilità del segnale PPS consente una sincronizzazione temporale precisa sull'intera popolazione di gateway in una rete. È un attivatore chiave per i beacon a livello di rete, che può essere utilizzato per la sincronizzazione dell'ora da parte dei dispositivi finali. La distribuzione PPS è attualmente supportata solo dalla stazione IoT Kerlink, dalla scheda di riferimento SX1301 e dall'IMST iC880A. \cite{art:rif.6}


\subsection{Applicazioni LoRa / Lv 7}
LoRa Technology sta connettendo il nostro pianeta intelligente. Esistono una moltitudine di applicazioni verticali IoT e implementate in tutto il mondo.
Un aspetto di notevole importanza per le applicazioni IoT è la criptazione di sicurezza incorporata nelle reti LoRa, che consente ai livelli applicazione e dispositivo di offrire una struttura di protezione dei dati personali o delle funzioni critiche dagli attacchi fisici o informatici. 
La questione della sicurezza della rete sta diventando sempre più importante, pertanto le reti LoRa richiedono elevati livelli di sicurezza.
Il raggiungimento dei necessari standard di sicurezza per le reti LoRa viene ottenuto impiegando diversi livelli di crittografia:
\begin{enumerate}
\item chiave univoca di rete (EUI64) che garantisce la sicurezza a livello di rete;
\item chiave unica di applicazione (EUI64) per garantire la sicurezza end-to-end a livello di applicazione;
\item chiave specifica del nodo finale (EUI128).
L’impiego di questi livelli di crittografia assicura che la rete LoRa rimanga sicura lungo tutto il percorso, dal nodo fino all’applicazione.
\end{enumerate}
Esistono due gruppi di domini crittografici in LoRa: il dominio di rete e il dominio dell'applicazione. Il dominio di rete è responsabile dell'autenticazione dei dati del nodo finale. La paternità di tali dati è verificata tramite una chiave segreta AES128 condivisa tra il dispositivo e il server di rete. Il dominio dell'applicazione è responsabile della garanzia della riservatezza dei dati del dispositivo. Esiste una chiave segreta AES128 condivisa tra l'applicazione utente e il nodo finale. LoRa è la soluzione più sicura disponibile sul mercato con crittografia 128AES su più livelli per tutti i dati dal sensore al server delle applicazioni e viceversa.


Il software Long Range Signaling and Control (LRSC) di IBM e al servizio sul cloud IBM Internet of Things Foundation, la LoRaWAN consente una facile implementazione su vasta scala di M2M/IoT. L'LRSC è il middleware o il collante che consente agli utenti di collegare, gestire e scalare milioni di dispositivi. IBM ha inoltre reso il protocollo LoRaWAN open source (Eclipse Public License) per lo sviluppo del nodo finale conosciuto come "LoRaWAN in C".

\subsection{Aspetti legislativi}
Nonostante la richiesta energetica contenuta, c'è ancora un possibile ostacolo. Anche se diversi Paesi europei stanno definendo il quadro legislativo per le reti LoRa, questi accordi non sono stati ancora conclusi e oggi le reti pubbliche non sono molto diffuse. In ogni caso si prevede che, una volta definiti gli aspetti legislativi, ci sarà una rapida accelerazione delle reti pubbliche LoRa. Verrà in tal modo stimolata la domanda di mercato per i dispositivi IoT Edge in tecnologia LoRa che, per il momento, non sono ancora pronti in quanto i produttori attendono il decollo di queste reti. Questa situazione sembrerebbe paradossale: solo pochi saranno capaci di iniziare lo sviluppo prima che le reti pubbliche siano disponibili, ma gli operatori di rete devono avere fiducia nel mercato per stare al passo con i tempi di introduzione delle leggi e impegnarsi nella realizzazione delle reti LoRa. La tecnologia LoRa possa offrire la connettività adatta alle applicazioni IoT asset-based, che dipendono dalle comunicazioni long-range a basso consumo ma, nello stesso tempo, richiedono la certezza che la rete sarà supportata per tutta la durata degli asset e dei relativi prodotti. L'aspetto più importante di questa situazione è il fatto che alcune aziende stanno pensando a un impiego nella propria rete privata LoRa, che permetterà ai clienti di connettere i dispositivi alla rete stessa per testare le loro applicazioni e definire i progetti pronti per la certificazione. Una volta completata l'attività legislativa oggi in atto in Europa, gli sviluppatori che avranno utilizzato questa rete privata per completare i loro progetti in conformità ai regolamenti approvati saranno fra i primi a presentare sul mercato i prodotti da utilizzare in rete.

% Certicazione dei prodotti LoRa Alliance Certified ™ %



\chapter{Cap3}

\chapter{Cap4}
TCP/IP over Avian Carriers\cite{waitzman1990standard}
\bibliographystyle{plain}
\bibliography{Biblio}
%\addcontentsline{toc}{chapter}{Bibliografia}
\end{document}
















\end{document}